\section{Introduction}\label{intro}
\subsection{Motivation}\label{intro:motivation}

With billions more IoT devices predicted to be connected every year,~\cite{gartner_billions_iot} practically every problem in IoT is realised at large scale. The connection of so many devices inevitably means that manufacturers iteratively improve their products; change and improve the deployed software. These rapid iterations may lead easily to unintended incompatibilities between server APIs, and those outdated or mis-configured APIs that client devices attempt to use. Additionally, many of the devices expected to be deployed will be deeply embedded in their operating context: light switches, concrete-borne traffic or structural sensors in roads or buildings,~\cite{construction_and_iot, smart_concrete} et cetera. Such devices have relatively long lifetimes and are rather easy to be forgotten about entirely compared to, say, an IoT teapot in a consumer's kitchen.

If client-originating communication with a server results in an error, this may be of great concern to the end-user, but might well present only to the manufacturer - or an owner separate to the concerned `user', as in traffic and structural sensors.

With the fairly common separation of software engineers developing back-end servers and those working on the IoT devices with which they will communicate, it is easy to imagine a scenario in which a small but important detail is miscommunicated: is the URI \code{/foobar} or \code{/foobars}; is the data returned in a flat JSON object, or is it nested under \code{data} with pagination? These are questions that the third-party or hobbyist developer, typically working against an existing system, is able to answer by manually inspecting it. The team working in tandem with the creation of the API however, is reliant on communication of these details - both in the first instance, and if there should be any later change. Even in the presence of good testing, such bugs can slip through.

Moreover, if requirements do change and the API is affected, then even if it is communicated to the team developing the client device, it may be a tediously large exercise to transform the data sent or modify the URI in many places.

\subsection{Requirements} \label{intro:req}

To obviate uncertainty surrounding the correctness of a client device's consumption of an API, we would like a system that leverages the type system and compiled nature of a language used for embedded systems, in order to provide certain compile-time validations.

There are many ways in which HTTP (or CoAP\footnote{CoAP is a stripped-down sibling of HTTP's, designed specifically for constrained IoT devices and networks.~\cite{rfc7252}}) interaction between a client and server can fail; indeed at all layers in the TCP/IP stack. It is the application layer that is of concern here, wherein failure modes are described by the \code{4xx} and \code{5xx} status code ranges, representing client and server errors respectively.

\subsubsection{Preventable client errors} \label{intro:req:preventable}
HTTP's client error status codes include `\code{405} Method Not Allowed', indicating that the method used (\code{GET}, \code{POST}, et al.) is not an action available for the resource on which it was attempted, and `\code{400} Bad Request', which is typically indicative of a malformed request body - such as \mintinline{json}`{"id": 42}` when the field \mintinline{json}{"id"} is expected to contain a string. Server error codes mostly pertain to ephemeral issues - unavailability due to the machine rebooting, or the web server having collapsed under sudden demand, for example. It is the possibility of client errors, the \code{4xx} range, that we can aim to avoid.

Thus, the system should aim to prevent the possibility, so far as possible, of errors occurring due to bugs in the client device code. Table~\cref{tbl:req:preventable-errors} details the codes it is expected to be possible to prevent.

\begin{table}[!h]
\caption{\emph{Client errors anticipated to be compile-time avoidable}}
\label{tbl:req:preventable-errors}
\centering
\begin{adjustwidth}{-1in}{-1in}
\begin{tabular}{lll}
\toprule
\multicolumn{1}{c}{\multirow{2}{*}{\textbf{Status Code}}} & \multicolumn{2}{c}{\textbf{Example}}
\\\cmidrule(lr){2-3}
& \multicolumn{1}{c}{\textbf{Use}} & \multicolumn{1}{c}{\textbf{Problem}}
\\\midrule
400 Bad Request & Request body has structure $A$ & Expected structure $B \neq A$
\\\cmidrule(lr){1-3}
401 Unauthorized & Unauthenticated request & Endpoint requires authentication
\\\cmidrule(lr){1-3}
404 Not Found & \code{GET /foo} & \code{/foo} does not exist
\\\cmidrule(lr){1-3}
405 Method Not Allowed & \code{PATCH /foo} & \code{/foo} does not support \code{PATCH}
\\\cmidrule(lr){1-3}
406 Not Acceptable & \code{Accept: application/json} & Endpoint does not provide JSON
\\\cmidrule(lr){1-3}
407 Proxy Auth. Required & Unauthenticated proxied request & Endpoint requires authentication
\\\cmidrule(lr){1-3}
411 Length Required & No \code{Content-Length} header & Endpoint requires the header
\\\cmidrule(lr){1-3}
412 Precondition Failed & \code{PATCH /foo \{"state": 3\}} & Endpoint cannot transition  $1 \to 3$
\\\cmidrule(lr){1-3}
415 Unsupported Media Type & \code{POST /foo \{\}} & Endpoint does not accept JSON
\\\cmidrule(lr){1-3}
417 Expectation Failed & \code{Expect: 100-continue} & Endpoint cannot respond \code{100}
\\\cmidrule(lr){1-3}
421 Misdirected Request & Connection to $A$ reused against $B$ & $B$ has greater security requirements
\\\cmidrule(lr){1-3}
424 Failed Dependency & \code{POST /dependant \{"ref": "foo"\}} & Dependency must be created first
\\\cmidrule(lr){1-3}
426 Upgrade Required & \code{HTTP/1.0 ...} & Endpoint requires \code{HTTP/1.1} 
\\\cmidrule(lr){1-3}
428 Precondition Required & \code{PUT ...} & Must \code{If-Match} on a priori state
\\\bottomrule
\end{tabular}
\end{adjustwidth}
\end{table}

\Cref{tbl:req:preventable-errors} forms a proper subset of possible client errors, since some such as \code{410 Gone} (which indicates a resource did exist, but no longer does), \code{403 Forbidden} (user has correctly authenticated, but is not authorised for this action), and \code{409 Conflict} (modification to the resource is in conflict with its a priori state) are caused by some aspect variable in the lifetime of the application.

Further, codes such as \code{409} and \code{410} could be due to the API interaction of other actors, if multiple devices have access to the same resource, in which case there is no hope to prevent the error entirely.

Others may be possible to prevent partially - such as the \code{422 Un\-proc\-ess\-able Entity} which typically signals a validation error within a (structurally sound, so \code{400} is inappropriate) request body, and those pertaining to data in the request being too long.

\subsubsection{Sessions} \label{intro:req:sessions}
Preventing some of the HTTP status codes listed in \cref{tbl:req:preventable-errors} will require knowledge of a `session' of multiple requests: validating that an authentication request has occurred prior to one that requires it, for example.

Thus, for maximum benefit, a solution will be required to analyse the stateful interaction of a series of requests, in addition to its analysis of each request in itself.

\subsubsection{Unpreventable client \& server errors} \label{intro:req:unpreventable}
Although we cannot hope - through any amount of analysis of client software - to stop server errors from occurring, it would be desirable to make the programmer aware of the possibility; to force him, via the compiler, to handle the error cases in a deliberate manner.

\subsubsection{Productivity} \label{intro:req:productivity}
A `softer', but no less important requirement is that the implementation allows the user (that is, the programmer of the client device software) to be productive. An implementation that burdens the user with additional work without perceivable gain would not be used; even if another party, perhaps the end-user of the device, would see the benefit.

Throughout the industry, burdensome technical advantage is traded off again\-st ease of development; the latter almost always wins. The result is that formally verified programs are mostly confined to research, and even compiled languages seem increasingly to be used only where necessary, with efforts to bring higher-level languages such as JavaScript to desktop platforms,~\cite{electron} and even to embedded devices.~\cite{jerryscript} Statically typed languages represent a cross-section that seems somewhat resistant to this trend, with projects such as the popular TypeScript~\cite{typescript} that transpiles to JS offering the boost to productivity that comes with the earlier warning of a large category of errors.

\begin{quote}
[Researchers in the 1980s] predicted that the programming world would embrace with gratitude every assistance promised by formalisation to solve the problems of reliability ... It has turned out that the world just does not suffer significantly from the kind of problem that our research was originally intended to solve.
\end{quote}\rightline{\rm --- Professor Sir Tony Hoare}

Thus, the implementation discussed here should be an aid, not a hindrance, to the user. It should make handling failure modes discussed in \cref{intro:req:unpreventable} straightforward, be transparent about why certain options are unavailable, and, in accordance with \cref{intro:motivation}, be of benefit to the development of the client device without impacting or requiring changes on any server with which it communicates.