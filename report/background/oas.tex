\subsection{Open API Specification} \label{bg:OAS}

The Open API Specification (OAS) is a work-in-progress evolution of the proprietary \emph{Swagger} specifications; volunteered by the company to form the \emph{Open API Initiative} (OAI) which aims to provide `a vendor neutral API description format'. \cite{about_oai}

The specification consists of multiple `objects' describing security, available paths, possible responses, the structure of requests, etc. - it is the `schema object' for the structure of request bodies that will be of most interest.

\begin{codelisting}{code:OAS:schema-obj}{Example of an OAS Schema Object}
\begin{minted}[tabsize=2]{yaml}
type: object
required:
- name
properties:
  name:
    type: string
  address:
    $ref: '#/definitions/Address'
  age:
    type: integer
    format: int32
    minimum: 0
\end{minted}
\end{codelisting}

For example, \cref{code:OAS:schema-obj} gives an OAS schema object that describes an object with at least a \code{name} field, and optionally an \code{address} and \code{age}. It also specifies that \code{name} is of type \code{string}, \code{age} is an unsigned $32$-bit integer, and the type of \code{address} is defined elsewhere in the document (in the `definitions object') to enable reuse.

\begin{codelisting}{code:OAS:path-obj}{Example of an OAS Path Object}
\begin{minted}[tabsize=2]{yaml}
/addresses:
  summary: Address book endpoint
  get:
    description: List addresses in user's address book
    responses:
      200:
        description: List of addresses
        content:
          application/json:
            schema:
              type: array
              items:
                $ref: '#/definitions/Address' 
  post:
    description: Add a new address
    requestBody:
      $ref: '#/definitions/Address'
    responses:
      405:
        description: Not implemented
/addresses/{address-id}:
  summary: Individual address
  get:
    description: Fetch a single address
    parameters:
    - name: address-id
      in: path
      description: ID of address
      required: true
      type: string
    responses:
      200:
        description: Address
        content:
          application/json:
            schema:
              $ref: '#/definitions/Address'
\end{minted}
\end{codelisting}

A path object, such as the two top-level entities in \cref{code:OAS:path-obj}, specifies the URI at which a particular endpoint is available, along with the HTTP methods that it accepts. For each endpoint, it then gives a description of the structure of expected requests, and that of the server's response.

An equivalent JSON document format is also offered, better demonstrating that the `schema object' in the OAS implements a subset of JSON Schema~\cite{oas_v3} - the JSON document structural validation language~\cite{json_schema} - and that the document itself is describable as a JSON Schema~\cite{oas_json_schema}.

JSON Schema also defines an extension - `Hyper-Schema' - that is similar in nature to the OAS, though strictly JSON, with different syntax, and restricted in scope. \cite{leach_elegant_apis_2014}

A third alternative, \emph{RAML}, is similar in feature set to the OAS but with a syntax differing enough to break some JSON Schema compatibility~\cite{raml_v1}. OAS also has the benefit of greater community use (inherited from \emph{Swagger}), and the advantage that comes from an open source initiative aiming for industry standard, as opposed to semi-open proprietary standard.