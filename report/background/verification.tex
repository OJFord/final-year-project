\subsection{Formal verification} \label{bg:verification}

\subsubsection{Design by contract} \label{bg:verification:dbc}

Hoare's logic introduced the `Hoare triple' construction, in which a valid pre-condition $P$ ensures that an expression or program $c$ will result in a valid post-condition $Q$. \cite{hoare_logic}

\begin{defn}Hoare triple \label{def:bg:verification:dbc:hoare-triple}
\[
    \left\{P\right\} c \left\{Q\right\}
\] If $P$ holds, then after evaluating $c$, $Q$ will hold true.
\end{defn}

`Design by contract' is the metaphorical term used to popularise the application of the Hoare triple to reason about the correctness of functions when programming.

For example, if we wish to chain the functions \code{foo} and \code{bar} as \code{bar(foo(42))} where we have that $\{P\}\operatorname{foo}\{Q\}$ and $\{U\}\operatorname{bar}\{V\}$, we must know that $Q$ is sufficient for $U$ in order to be able to say that $\{P\} \operatorname{bar}\left(\operatorname{foo}\left(42\right)\right) \{V\}$.

The `contract' is figuratively entered into by the caller of a function or method, and the implementer or the function itself. Providing the caller meets a set of `terms' - the precondition - the function will also adhere to its terms - viz. the post-condition.

\emph{rustproof} is a \emph{rust} compiler plugin \cfref{bg:rust:plugins} which consumes pre- and post-condition annotations for users' function, and using Dijkstra's predicate transformer semantics generates weakest liberal pre-conditions from the annotations and function body. The SMT solver Z3 is then used to verify the condition $P \implies \operatorname{wlp(S, Q)}$, where $P,Q$ are the supplied conditions, $\operatorname{wlp}$ the weakest pre-condition, and $S$ is the set of statements forming the function body. \cite{rust_rustproof}

\subsubsection{Coq proof assistant} \label{bg:verification:coq}

An extension to Hoare logic~\cfref{bg:verification:dbc} incorporating a predicate to hold even mid-execution of $c$ is `Crash Hoare Logic'; (CHL) in the work inspiring its formulation, the authors developed a functional UNIX filesystem (FSCQ) in \emph{Coq} that is proven to operate correctly if there are no crashes, and also guaranteed to recover from a crash in a consistent state if it does indeed recover at all. \cite{fscq} (No amount of formal verification can prevent a user rebooting the system ad infinitum!)

To some extent, FSCQ challenges the claim in~\cref{intro:req:productivity}, that formal verification is too complex and time-consuming for many real-world applications: the six authors implemented the system in around two years, including the development of reusable components like CHL for \emph{Coq}~\cite{fscq}. In contrast, APFS was recently released by \emph{Apple} after nearly four years' development;~\cite{apfs_detail} and likely by a considerably larger team.

There remains a significant barrier to entry though; that the FSCQ project is exceptional in spanning both the theoretical and practical disciplines so effectively.~\cite{soft_eng_possible} \Cref{code:coq:fsm} gives an example of a finite state machine in \emph{Coq} modelling the days of the week. There is a fair amount of ceremony involved in even the trivial proof that following the FSM edges (given by the function \code{nextday}\mintinline{coq}{:} \code{weekday} \mintinline{coq}{->} \code{weekday}) brings around to the starting point in exactly seven steps.

\begin{codelisting}{code:coq:fsm}{A simple proof in Coq: week days come around again every seven.}
\begin{spacing}{\codespacing}
\begin{minted}[tabsize=4]{coq}
    Inductive weekday: Set :=
        | monday: weekday
        (* ... *)
        | sunday: weekday.
    
    Function nextday (day: weekday) : weekday :=
        match day with
        | monday => tuesday
        (* ... *)
        | sunday => monday
        end.
    
    Function trans (day: weekday) (c: nat) : weekday :=
        match c with
        | 0 => day
        | S r => nextday (trans day r)
        end.
    
    Lemma modular: forall day: weekday, trans day 7 = day.
    Proof.
        intros.
        simpl.
        induction day.
        - reflexivity. (* Monday *)
        (* ... *)
        - reflexivity. (* Sunday *)
    Qed.
\end{minted}
\end{spacing}
\end{codelisting}

Considering that the proof in~\cref{code:coq:fsm} neglects the possibility that \code{trans\- day\- n\- =\- day} for some \mintinline{coq}{n < 7}, and probably many other issues making it unsuitable for an ultra-paranoid calendar application, this does not bode well for developer-friendly formally verified programming outside of safety-critical applications.

Further, even once a \emph{Coq}-assisted proof is complete, we still have an extra build step in order to `extract' the target language - ML variants, Haskell, etc. - before we can compile as usual.~\cite{coq_extraction} Without this step, and perhaps hand-written source code that is believed to be equivalent, the \emph{Coq} proof only tells us that a correct solution exists; not that we have it.

Perhaps the day will come when formal verification is sufficiently straightforward to be worthwhile for `every-day' calendar applications and the like. For the purposes of this project, though, some sacrifice of formalism - and thus of \emph{proof} - seems well justified for the corresponding gain in developer productivity.

\begin{quote}
[Researchers in the 1980s] predicted that the programming world would embrace with gratitude every assistance promised by formalisation to solve the problems of reliability ... It has turned out that the world just does not suffer significantly from the kind of problem that our research was originally intended to solve. \cite{hoare_quote}
\end{quote}\rightline{\rm --- Professor Sir Tony Hoare}