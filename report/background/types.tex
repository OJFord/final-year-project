\subsection{Type theory}\label{bg:types}

\subsubsection{$\lambda$-calculus}\label{bg:types:lambda-calculus}
The untyped lambda calculus was introduced by Church in 1936; the set of terms $\Lambda$ is ranged over by $M$ as: $$
M \Coloneqq x \mid (\lambda{x}.M') \mid (M' \cdot M'')
$$ where $x$ is a variable; with parentheses and application operator ($\cdot$) often omitted. \cite{tsfpl}

\begin{defn}$\alpha$-conversion\label{def:types:alpha-conversion}
	$$
	\lambda{x}.M =_{\alpha} \lambda{y}.(M[y/x]) \ \ \ (y\notin{M})
	$$
\end{defn}

That is, terms differing only in the names of bound variables are ($\alpha$-)equivalent.

\begin{defn}Term substitution\label{def:types:term-substitution}
	$$
	\begin{aligned}
		x[N/x] &= N \\
		y[N/x] &= y &(y \neq x) \\
		(\lambda{y}.M) &= \lambda{y}.(M[N/x]) \\
		(PQ)[N/x] &= P[N/x]Q[N/x]
	\end{aligned}
	$$
\end{defn}

\begin{defn}$\beta$-reduction\label{def:types:beta-conversion}
	$$
	\begin{aligned}
		(\lambda{x}.M)N &\to_{\beta} M[N/x] \\
		M \to_{\beta} N &\Rightarrow \begin{cases}
 			\lambda{x}.M \to_{\beta} \lambda{x}.N \\
 			PM \to_{\beta} PN \\
 			MP \to_{\beta} NP
 		\end{cases}
	\end{aligned}
	$$
\end{defn}

$\beta$-conversion is a binary relation on $\langle{M,N}\rangle$, where $M$ is a \emph{reducible expression} (i.e. has the form $PQ$ where $P$ is an abstraction) and $N$ is its \emph{contractum}, the result of applying term substitution to $M$. The `many-step' relation $\twoheadrightarrow_{\beta}$ is its reflexive and transitive closure. \cite{tsfpl}

\begin{defn}$\eta$-reduction\label{def:types:eta-reduction}
	$$
	\begin{aligned}
	\lambda{x}.Mx &\to_{\eta} M &(x\notin{fv(M)})
	\end{aligned}
	$$
\end{defn} where $fv(M)$ denotes the free variables of $M$.

\paragraph{Extending $\Lambda$:}\label{bg:types:extending-lambda}

$\Lambda$ is often used as a basis for reasoning about programming languages, on which more and higher-level features can be built.~\cite{tpl} For example, adding explicit substitution and garbage collection: $$
M \Coloneqq \dots \mid x\langle{x \coloneqq N}\rangle
$$ then, similarly to $\beta$-reduction, the one-step and contextual closure:~\cite{lambda_xgc, tsfpl}
$$
\begin{aligned}
	&(\textrm{xv})\colon &x\langle{x \coloneqq N}\rangle &\to_{\textrm{xgc}} N \\
	&(\textrm{xvgc})\colon &x\langle{y \coloneqq N}\rangle &\to_{\textrm{xgc}} x &(y \neq x) \\
	&(\textrm{xab})\colon &(\lambda{x}.M)\langle{y \coloneqq N}\rangle &\to_{\textrm{xgc}} \lambda{x}.(M\langle{y \coloneqq N}\rangle) \\
	&(\textrm{xap})\colon &(MM')\langle{y \coloneqq N}\rangle &\to_{\textrm{xgc}} (M\langle{y \coloneqq N}\rangle)(M'\langle{y \coloneqq N}\rangle) \\
	&(\textrm{gc})\colon &M\langle{x \coloneqq N}\rangle &\to_{\textrm{xgc}} M &\big(x \notin fv(M)\big)
\end{aligned}
$$$$
\begin{aligned}
	M \to_{\textrm{xgc}} N \Rightarrow \begin{cases}
		ML &\to_{\textrm{xgc}} NL \\
		LM &\to_{\textrm{xgc}} LN \\
		\lambda{x}.M &\to_{\textrm{xgc}} \lambda{x}.N \\
		M\langle{x \Coloneqq L}\rangle &\to_{\textrm{xgc}} N\langle{x \Coloneqq L}\rangle \\
		L\langle{x \Coloneqq M}\rangle &\to_{\textrm{xgc}} L\langle{x \Coloneqq N}\rangle
	\end{cases}
\end{aligned}
$$

\subsubsection{Curry type assignment}\label{bg:types:curry}
The set of types $\mathcal{T}_C$ is ranged over by $A$ as: $$
A \Coloneqq \varphi \mid (A \to B)
$$ where $\varphi$ ranges over the set of type variables $\Phi$. \cite{tsfpl}

A \emph{statement} is of the form $M\colon{A}$, where $M$ is the subject and $A$ the predicate. A set of statements forms the context $\Gamma$, for which $x\in{\Gamma}$ iff $\exists A$ such that $x\colon{A}\in\Gamma$.

\begin{defn}Curry type assignment - derivations\label{def:types:curry-derivations}
	$$
	\begin{aligned}
		&(\textrm{Ax}): \Inf{
			\Gamma,x\colon{A} \vdash_C x\colon{A}
		} \\
		\\
		&(\to\textrm{I}): \Inf[x\notin{\Gamma}]{
			\Gamma,x\colon{A} \vdash_C M\colon{B}
		}{
			\Gamma \vdash_C \lambda{x}.M\colon{A\to{B}}
		} \\
		\\
		&(\to\textrm{E}): \Inf{
			\Gamma \vdash_C M\colon{A\to{B}} \ \ \ \Gamma \vdash_C M'\colon{A}
		}{
			\Gamma \vdash_C MM'\colon{B}
		}
	\end{aligned}
	$$
\end{defn}
Where $\Gamma,x\colon{A}$ is written to mean $\Gamma\cup\{x\colon{A}\}$ where either $x\colon{A}\in\Gamma$ or $x\notin\Gamma$.

\begin{defn}Subject reduction\label{def:types:subject-reduction}
	$$
	(\Gamma \vdash_C M\colon{A}) \;\&\; (M \twoheadrightarrow_{\beta} N)
	\;\implies\;
	\Gamma \vdash_C N\colon{A}
	$$
\end{defn}

\begin{defn}Type substitution $S = (\varphi\mapsto{C})\colon{\mathcal{T}_C\to\mathcal{T}_C}$\label{def:types:substitution}
	$$
	\begin{aligned}
		&(\varphi\mapsto{C})\varphi &= \ &C
		\\
		&(\varphi\mapsto{C})\varphi' &= \ &\varphi' \ &\ (\varphi'\neq\varphi)
		\\
		&(\varphi\mapsto{C})A\to{B} &= \ &((\varphi\mapsto{C})A)\to((\varphi\mapsto{C})B)
	\end{aligned}
	$$$$
	\begin{aligned}
		S \circ S' &= S(S'A) \\
		S\Gamma &= \{x\colon{SB} \mid x\colon{B}\in\Gamma\}
	\end{aligned}
	$$
\end{defn}
Type substitution is provably sound and complete, by induction on the structure of derivations (cf. Definition~\cref{def:types:curry-derivations}). \cite{tsfpl}

As in the untyped $\Lambda$, Curry's system gives a good basis on which to build further abstraction, such as Boolean or variant types. \cite{tpl}

\subsubsection{Algebraic types} \label{bg:types:algebraic}

!TODO

\subsubsection{$\pi$-calculus and session types} \label{bg:types:session}

!TODO