With billions more IoT devices predicted to be connected every year, \cite{gartner_billions_iot} practically every problem in IoT is realised at large scale. The connection of so many devices inevitably means that manufacturers iteratively improve their products; change and improve the deployed software. These rapid iterations may lead easily to unintended incompatibilities between server APIs, and those outdated or mis-configured APIs that client devices attempt to use. Additionally, many of the devices expected to be deployed will be deeply embedded in their operating context: light switches, concrete-borne traffic or structural sensors in roads or buildings, \cite{construction_and_iot, smart_concrete} et cetera. Such devices have relatively long lifetimes and are rather easy to be forgotten about entirely compared to, say, an IoT teapot in a consumer's kitchen.

If client-originating communication with a server results in an error, this may be of great concern to the end-user, but might well present only to the manufacturer - or an owner separate to the concerned `user', as in traffic and structural sensors.

With the fairly common separation of software engineers developing back-end servers and those working on the IoT devices with which they will communicate, it is easy to imagine a scenario in which a small but important detail is miscommunicated: is the URI \code{/foobar} or \code{/foobars}; is the data returned in a flat JSON object, or is it nested under \code{data} with pagination? These are questions that the third-party or hobbyist developer, typically working against an existing system, is able to answer by manually inspecting it. The team working in tandem with the creation of the API however, is reliant on communication of these details - both in the first instance and if there should be any later change. Even in the presence of good testing, such bugs can slip through.

Moreover, if requirements do change and the API is affected, then even if it is communicated to the team developing the client device, it may be a tediously large exercise to transform the data sent or modify the URI in many places.
